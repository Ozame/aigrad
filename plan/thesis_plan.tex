\documentclass[12pt]{article}

\begin{document}

\title{Master's thesis research plan}
\author{Samu Kumpulainen}
\date{\today}
\maketitle

\newpage

\section{Introduction}
1. Johdanto
- tutkimuksen idea, miksi se on tärkeä sekä tieteellisesti että käytännössä
\section{Literature mapping}
2. Kirjallisuuskartoitus 
- menetelmä: voidaan esitellä hakusanat, hakuprosessi, hakukoneet ja tietokannat (hyödyllistä tietoa kirjata itselle muistiin, ei välttämättä tule lopulliseen graduun)
- tulokset (tiivistetty kuvaus löytyneistä artikkeleista)
- kerrotaan, mitä tutkittavasta aiheesta tiedetään entuudestaan
metodilähteitä mainittava
\section{Research topic/question}
3. Tutkimusaihe/tutkimuskysymys
\section{Research method and reasoning}
4. Tutkimusstrategia/metodi ja sen valintaperusteet
\section{Material gathering plan}
5. Aineiston keruun suunnittelu
- mitä, keneltä, milloin, miten
- ml. eettiset näkökohdat (hyvä tieteellinen käytäntö jota noudatetaan, tietosuoja, tieteellisen tutkimuksen rekisteriseloste)
\section{Material gathering}
6. Aineiston keruu 
- kuvaa konkreetilla tasolla miten ja milloin aineisto kerätään, käsitellään, talletetaan, arkistoidaan/hävitetään
\section{Material analysis}
7. Aineiston analyysi 
- analyysin kuvaus, millä menetelmällä analyysi tehdään
\section{Results}
8. Tulokset
\section{Conclusion}
9. Johtopäätökset
\section{References}
10. Lähdeluettelo
\section{Appendix}
11. Liitteet

\end{document}