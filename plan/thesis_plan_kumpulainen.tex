\documentclass[utf8,english]{gradu3}

\usepackage{graphicx} % for including pictures

\usepackage{amsmath} % useful for math (optional)

\usepackage{booktabs} % good for beautiful tables

% NOTE: This must be the last \usepackage in the whole document!
\usepackage[bookmarksopen,bookmarksnumbered,linktocpage]{hyperref}

\addbibresource{plan_references.bib} % The file name of your bibliography database

\begin{document}

\title{Master's thesis research plan}
\translatedtitle{Pro Gradu -tutkimussuunnitelma}
\studyline{Mathematical Information Technology}
\avainsanat{%
  Pro Gradu,
  tutkielma}
\keywords{Master's Theses}
\tiivistelma{%
  
  Tässä suunnitelmassa käydään läpi pro gradu -tutkielman mahdollista aihetta ja tutkimustapaa.
}
\abstract{%
  This research plan contains description of the planned topic of master's thesis, its background, possible research methods and approaches.
}

\author{Samu Kumpulainen}
\contactinformation{\texttt{samu.p.kumpulainen@student.jyu.fi}}
% use a separate \author command for each author, if there is more than one
\supervisor{Vagan Terziyan}
% use a separate \supervisor command for each supervisor, if there
% is more than one

 % you don't need this line in a thesis
\maketitle

\mainmatter

%Remember to use chapters in the thesis itself
\section{Introduction}
The thesis will perform a systematic research mapping on the field of Artificial General Intelligence (AGI)). The goal of the study is to identify the themes and subfields of AGI research in recent years, what is being researched currently, and what kind of gaps exist on the field. For a while the AGI field was not so active, the more specific approaches, 'narrow AI', grew in popularity. Recently, however, the call for a wider, more general artificial intelligence has been regaining interest. This kind of research mapping study would be needed as the research field is complex and there is no clear presentation of the current trends and focal points. 


\section{Literature mapping} 
Systematic mapping study creates an overview of the research area by categorizing the reports and studies, creating a visual map that provides information on how the research is focused. The method is based on observing the abstracts of studies, enabling faster analysis and greater volume in the material. In comparison, systematic literature review method goes on more detail, providing more verbal, summary-like results on the topic. 

\subsection{Possible material sources}
The material is to be gathered through databases and search engines via specified search terms. Databases and content libraries such as IEEE, ACM, and Google Scholar can be used. One possible option would be to focus on journals that specialize on the field, such as \textit{Journal of Artificial General Intelligence, Journal of Artificial Intelligence Research, and Artificial Intelligence}, the first of which is highly focused on the area, but not well ranked based on Publication Forum. 
\subsection{Method sources}

There exists many good papers on the research method. There exists some good example theses using the approach, such as Mononen() and Ryynänen(). Guidelines regarding the method itself can be found on Peterson and ...
- menetelmä: voidaan esitellä hakusanat, hakuprosessi, hakukoneet ja tietokannat (hyödyllistä tietoa kirjata itselle muistiin, ei välttämättä tule lopulliseen graduun)
- tulokset (tiivistetty kuvaus löytyneistä artikkeleista)
- kerrotaan, mitä tutkittavasta aiheesta tiedetään entuudestaan
metodilähteitä mainittava


\section{Research topic/question}
The research questions are not too specific yet, but some ideas are:
\begin{itemize}
    \item What is the current state of AGI?
    \item What are the current techniques used and researched?
    \item What have been the most successful attempts recent years?
    \item How has the generalization of specific techniques advanced recently?
\end{itemize}


\section{Research method and reasoning}
The research method, systematic mapping, is a secondary study method that aims to create a general view of the studied area by a process of keywording, classification, and mapping. 

\section{Material gathering plan}


- mitä, keneltä, milloin, miten
- ml. eettiset näkökohdat (hyvä tieteellinen käytäntö jota noudatetaan, tietosuoja, tieteellisen tutkimuksen rekisteriseloste)
\section{Material gathering}


- kuvaa konkreetilla tasolla miten ja milloin aineisto kerätään, käsitellään, talletetaan, arkistoidaan/hävitetään
\section{Material analysis}


- analyysin kuvaus, millä menetelmällä analyysi tehdään
\section{Results}


\section{Conclusion}

Johtopäätökset
\section{References}


\section{Appendix}


\end{document}