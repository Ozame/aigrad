\documentclass[utf8,english]{gradu3}

\usepackage{graphicx} % for including pictures

\usepackage{amsmath} % useful for math (optional)

\usepackage{booktabs} % good for beautiful tables

\usepackage{csquotes} % ensure proper formatting

\usepackage[authordate,backend=biber,noibid]{biblatex-chicago}

% NOTE: This must be the last \usepackage in the whole document!
\usepackage[bookmarksopen,bookmarksnumbered,linktocpage]{hyperref}

\addbibresource{thesisReferences.bib} % The file name of your bibliography database

\begin{document}

\title{Master's thesis research plan}
\translatedtitle{Pro Gradu -tutkimussuunnitelma}
\studyline{Mathematical Information Technology}
\avainsanat{%
  Pro Gradu,tutkielma, AI, tekoäly}
\keywords{Master's Theses, AGI, AI, artificial intelligence}
\tiivistelma{%
  
  Tässä suunnitelmassa käydään läpi pro gradu -tutkielman mahdollista aihetta ja tutkimustapaa.
}
\abstract{%
  This research plan contains description of the planned topic of master's thesis, its background, possible research methods and approaches.
}

\author{Samu Kumpulainen}
\contactinformation{\texttt{samu.p.kumpulainen@student.jyu.fi}}
% use a separate \author command for each author, if there is more than one
\supervisor{Vagan Terziyan}
% use a separate \supervisor command for each supervisor, if there
% is more than one

\maketitle

\mainmatter

%Remember to use chapters in the thesis itself
\section{Introduction}
The thesis will be a systematic research mapping on the field of Artificial General Intelligence (AGI). The goal of the thesis is to identify the themes and subfields of AGI research in recent years, what is being researched recently, and what kind of gaps exist on the field. For a while the AGI field was not so active and the more specific approaches, 'narrow AI', grew in popularity. Recently, however, the wider, more general artificial intelligence has been regaining interest. This kind of mapping study would be needed as the research field is complex and there is no clear presentation of the current trends and focal points. Creating this king of overview would be a valuable asset for future research. Furthermore, if an interesting subtopic comes up during the process of mapping, more focus may be directed towards that in form of more traditional systematic literature review. This option is left for further consideration.


\section{Literature mapping} 
As the thesis itself will be a systematic literature mapping study, no separate review or mapping should be needed. Systematic mapping study creates an overview of the research area by categorizing the reports and studies, creating a visual map that provides information on how the research is focused. The method is based on observing the abstracts of studies, enabling faster analysis and greater volume in the material. It is described in more detail in \ref{method}. In comparison, systematic literature review method goes on more detail, providing more verbal, summary-like results on the topic. This increased depth narrows the breadth of the study. 

\subsection{Possible material sources}
\label{mat_sources}
The material is to be gathered through databases and search engines via specified search terms. Databases and content libraries such as IEEE, ACM, and Google Scholar can be used. One possible option would be to focus on journals that specialize on the field, such as \textit{Journal of Artificial General Intelligence, Journal of Artificial Intelligence Research}, and \textit{Artificial Intelligence}, the first of which is highly focused on the area, but not well ranked based by Publication Forum. 
\subsection{Method sources}

There exists many good papers on the research method. There are some example theses using the approach, such as Niko Mononen's (\cite{mononen2018}) and Sari Ryynänen's (\cite{ryynanen2017}). Guidelines regarding the method itself can be found on article ny Petersen et al. \cite*{petersen2008}. There is an update on the topic as well (\cite{petersen2015}). Some sources about literature reviews such as Bereton et al. (\cite*{brereton2007}) and Salminen (\cite*{salminen2011}) might be useful as well.

\subsection{Previous knowledge}

The topic of artificial general intelligence is somewhat familiar to me, as it has been touched on courses such as Collective Intelligence and Agent Technology, and Deep Learning for Cognitive Computing. The field is complex and has a lot of history, and I am certain that it has a lot to offer in the future. One reason for the complexity is the fact that there are many different approaches and considerations both in theory and in practice. There will be a lot purely theoretical studies varying from the possible frameworks and technologies used in development to robot ethics and machine morality.


\section{Research topic/question}
The research questions are not too specific yet, but some ideas are:
\begin{itemize}
    \item What is the current state of AGI?
    \item What are the current techniques used and researched?
    \item What have been the most successful attempts recent years?
    \item How has the generalization of specific techniques advanced recently?
\end{itemize}


\section{Research method and reasoning}
\label{method}
The research method, systematic mapping, is a secondary study method that aims to create a general view of the studied area by a process of keywording, classification, and mapping. According to Peterson et al. (\cite*{petersen2008}), it consist of the following phases:
\begin{enumerate}
    \item Definition of research questions
    \item Conducting search
    \item Screening the papers for inclusion and exclusion
    \item Keywording (abstracts, conclusions, introductions)
    \item Data extraction and mapping
\end{enumerate}

Each phase produces a subresult to be used in the next one. This process results in a systematic map of the area. This can and should be further visualized using for example bubble graphs (\cite{mononen2018} and \cite{petersen2008}). This helps to more easily spot the focus points and gaps in the research.

\section{Material gathering plan}
The material gathering will be performed during the year 2020, when the actual thesis writing takes place. As mentioned in \ref{mat_sources}, the data will be gathered through the internet databases. The criteria for the inclusion of papers need to be thoroughly considered, for example they need to be available in English and accessible without charge.

As the study method will be a secondary study, and the data used will be public research performed by other people, the ethical considerations will consist only on the data access rights. The ethics of the research used as data are not the concern of this thesis.

\section{Material gathering}
During the material gathering phase, as it is found on the databases, it might be wise to download and store it locally, if possible. This would make the analysis itself easier to keep track of. In any case, a process needs to be put into place to keep track which studies are waiting to examined, which have been processed etc.
As a secondary study, no separate archiving and disposal procedures are necessary. The results and notes on the thesis itself will be saved using version control such as Git.
\section{Material analysis}

The material will be analysed as presented in \ref{method}, with process of keywording, abstraction and schema building. Further synthesis can be derived from the results of this method.
\section{Results}
The results of the thesis will be a clear overview of the recent research in the field of artificial general intelligence in form of classification data, visual graphs and further synthesis. As mentioned earlier, if an interesting topic presents itself during the process of mapping, it may be further examined in a more focused way, such as literature review. This will be considered at that time.

\section{Conclusion}
In short, a systematic mapping will be performed on the field of Artificial General Intelligence to achieve a clear overview of the field. This can be used in further research to see which areas need to be examined in more detail.


\printbibliography


\end{document}