\documentclass[utf8,english]{gradu3}

\usepackage{graphicx} % for including pictures

\usepackage{amsmath} % useful for math (optional)

\usepackage{booktabs} % good for beautiful tables

\usepackage{csquotes} % ensure proper formatting

\usepackage[authordate,backend=biber,noibid]{biblatex-chicago}

% NOTE: This must be the last \usepackage in the whole document!
\usepackage[bookmarksopen,bookmarksnumbered,linktocpage]{hyperref}

\addbibresource{thesisReferences.bib} % The file name of your bibliography database

\begin{document}

\title{Artificial General Intelligence - a systematic mapping study}
\translatedtitle{Yleistekoäly - systemaattinen kirjallisuuskartoitus}
\studyline{Mathematical Information Technology}
\avainsanat{%
  Pro Gradu,tutkielma, AI, tekoäly}
\keywords{Master's Theses, AGI, AI, artificial intelligence, systematic literature mapping, mapping study}
\tiivistelma{%
  
  Tässä suunnitelmassa käydään läpi pro gradu -tutkielman mahdollista aihetta ja tutkimustapaa. TODO: translate when abstract done
}
\abstract{%
  In this thesis, a systematic mapping study is performed on the field of artificial general intelligence. The goal of the study is to gain insight about the recent developments in the study field. This includes the focus points of the current reserch, possible research gaps, and how the research itself is conducted. TODO: more accurate, proper abstract
}

\author{Samu Kumpulainen}
\contactinformation{\texttt{samu.p.kumpulainen@student.jyu.fi}}
% use a separate \author command for each author, if there is more than one
\supervisor{Vagan Terziyan}
% use a separate \supervisor command for each supervisor, if there
% is more than one

\maketitle

\mainmatter

%Remember to use chapters in the thesis itself
\chapter{Introduction}
The thesis will be a systematic research mapping on the field of Artificial General Intelligence (AGI). The goal of the thesis is to identify the themes and subfields of AGI research in recent years, what is being researched recently, and what kind of gaps exist on the field. For a while the AGI field was not so active and the more specific approaches, 'narrow AI', grew in popularity. Recently, however, the wider, more general artificial intelligence has been regaining interest. This kind of mapping study would be needed as the research field is complex and there is no clear presentation of the current trends and focal points. Creating this king of overview would be a valuable asset for future research, as it would enable focusing the research on areas less ventured. Furthermore, if an interesting subtopic comes up during the process of mapping, more focus may be directed towards that in form of more traditional systematic literature review. This option is left for further consideration.



\chapter{Artificial General Intelligence} 



\section{Definition}

% Definition of intelligence?
In order to be able to define AGI, or artificial intelligence in general, one must first consider the definitions of intelligence in general. The exists many different definitions, in many different branches of science. Legg and Hutter (\cite*{legg2007}) list over 60 definitions collected from various academic sources. These include, for example, \emph{"the general mental ability involved in calculating, reasoning, perceiving relationships and analogies, learning quickly, storing and retrieving information,using language fluently, classifying, generalizing, and adjusting to new situations."} (Columbia Encyclopedia, sixth edition, 2006), \emph{"that facet of mind underlying our capacity to think, to solve novel problems, to reason and to have knowledge of the world"} (\cite{anderson2006}), and \emph{"Intelligence is the ability for an information processing system to adapt to its environment with insufficient knowledge and resources."} (\cite{wang1995}).
Based on the aforementioned collection of definitions, Legg and Hutter (\cite*{legg2007}) have formed the following definition: \emph{"Intelligence measures an agent's ability to achieve goals in a wide range of environments"}. This gives us a single definition which encompasses the common traits in intelligence definitions. 
% Measurement? or just later when roadmap is discussed?
% in any case, needs something more here, now just quotes :(

%Intelligence is a trait that manifests itself in countless ways, making a single common definition a challenge. B 

Artificial General Intelligence, sometimes referred as "strong ai", according to Goertzel and Pennachin (\cite*{goertzel2007}) means \emph{"AI systems that possess a reasonable degree of self-understanding and autonomous self-control, and have the ability to solve a variety of complex problems in a variety of contexts, and to learn to solve new problems that they didn't know about at the time of their creation."}. It is abbreviated as AGI. The reason general intelligence is specified instead of plain intelligence is that there is a need to differentiate it from the domain specific artificial intelligence, also known as "narrow AI" or "weak AI". Terms strong AI and weak AI were coined by John Searle in 1980 (\cite{searle1980}). 


% unclear, discussion about the meaning? some other sources here

\section{History of AGI} 
% More about the origins of AI here. 
% 
Ever since the Artificial Intelligence research entered the academia, speculation about an AI that would reach the human level intelligence has been surfacing every few years. For example, American AI scientist Marvin Minsky suggested in 1970 that a machine with human level general intelligence could be developed in the next few years (\cite{kaplan2019}). This never happened, and soon after that whole AI research lost its interest in the eyes of the financial backers. The time period following this is known as the first "AI winter" (\cite{kaplan2019}).

% here about the landscape and efforts of trying to find common ground

TODO: What next? ups, downs, then narrow ai settling in, AGI being forgotten for a while, until returning later on?


\chapter{Systematic literature mapping process}

\section{Research method and reasoning}

\label{method}
The research method, systematic mapping, is a secondary study method that aims to create a general view of the studied area by a process of keywording, classification, and mapping. According to Peterson et al. (\cite*{petersen2008}), it consist of the following phases:
\begin{enumerate}
    \item Definition of research questions
    \item Conducting search
    \item Screening the papers for inclusion and exclusion
    \item Keywording (abstracts, conclusions, introductions)
    \item Data extraction and mapping
\end{enumerate}

Each phase produces a subresult to be used in the next one. This process results in a systematic map of the area. This can and should be further visualized using for example bubble graphs (\cite{mononen2018} and \cite{petersen2008}). This helps to more easily spot the focus points and gaps in the research.


The material is to be gathered through databases and search engines via specified search terms. Databases and content libraries such as IEEE, ACM, and Google Scholar can be used. One possible option would be to focus on journals that specialize on the field, such as \textit{Journal of Artificial General Intelligence, Journal of Artificial Intelligence Research}, and \textit{Artificial Intelligence}, the first of which is highly focused on the area, but not well ranked based by Publication Forum. 

There exists many good papers on the research method. There are some example theses using the approach, such as Niko Mononen's (\cite{mononen2018}) and Sari Ryynänen's (\cite{ryynanen2017}). Guidelines regarding the method itself can be found on article ny Petersen et al. \cite*{petersen2008}. There is an update on the topic as well (\cite{petersen2015}). Some sources about literature reviews such as Bereton et al. (\cite*{brereton2007}) and Salminen (\cite*{salminen2011}) might be useful as well.

\section{Difference with other secondary studies}

\section{Mapping studies in field of IT}

\chapter{The literature mapping}

\section{Background/why this method and topic}
The research questions are not too specific yet, but some ideas are:
\begin{itemize}
    \item What is the current state of AGI?
    \item What are the current techniques used and researched?
    \item What have been the most successful attempts recent years?
    \item How has the generalization of specific techniques advanced recently?
\end{itemize}

\section{Research questions}

The following research questions...


\begin{enumerate}
  \item How much, and what type of research is done in the field of AGI?
  \item Where is the AGI research focused on?
  \item Has there been any major breakthroughs?
  \item Where and when were the studies published?
\end{enumerate}

\section{Sources and databases used}
- Journal listing and their date ranges etc.

- search terms here or another section? 

- table showing used search phrases?

\section{conducting search}
search phrases are used on different databases, limiting the papers to amount possible to handle

\section{Criteria for inclusion, exclusion}
papers from the 

\section{Keywording}
papers are further analyzed, keywords are extracted from abstracts, 

\section{Data extraction and mapping}
keywords are mapped using frequencies etc

\section{Source material control}
- How the papers were handled
- How graphs etc. were made

\chapter{Results and analysis}
The results of the thesis will be a clear overview of the recent research in the field of artificial general intelligence in form of classification data, visual graphs and further synthesis. As mentioned earlier, if an interesting topic presents itself during the process of mapping, it may be further examined in a more focused way, such as literature review. This will be considered at that time.

\section{Results of literature mapping}
- Graphs and other visualization, bubble graphs are useful.

\section{Possible continuation research}

- List of most prominent topics for further research

\chapter{Conclusion}
In this thesis, a systematic literature mapping was conducted on the field of artificial general intelligence. Results of the study showed that .... 

\printbibliography


\end{document}